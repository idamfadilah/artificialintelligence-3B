\chapter{Kecerdasan buatan}
\section{TEORI}
\begin{enumerate}
	\item Definisi\\
	Kecerdasan buatan atau sering dikenal dengan AI (artificial intelligence) adalah model proses proses manusia berpikir yang di desain dalam suatu mesin agar bisa meniru perilaku layaknya manusia
	\item Sejarah\\
    LISP merupakan Bahasa pemrograman tingkat tinggi yang diciptakan oleh John McCarthy pada tahun 1958, dimana Bahasa ini mendominasi pembuatan program-program kecerdasan buatan. Setelah LISP, McCarthy membuat program yan dinamakan Program with Common Sense, pada program ini McCarthy merancang suatu teknologi yang berfunsi sebagai pemecahan masalah. Di tahun 1959, Nathaniel Rochester dari IBM juga mahasiswa-mahasiswanya merilis sebuah program kecerdasan buatan yaitu Geometry Theorm Prover, yang mana program ini digunakan untuk menciptakan suatu teorema dengan memakai pertanyaan – pertanyaan yang tersedia. Namun pada sekitar tahun 1960 – 1970 perkembangan AI sempat melambat. Hal ini dikarenakan program-program kecerdasan buatan yang bermunculan hanya mengandung sedikit atau bahkan tidak mengandung sama sekali pengetahuan pada subjeknya. Produk kecerdasan yang sukses digunakan adalah hanya sebatas hasil manipulasi sederhana.\\
    Pada tahun 1980 Ai mulai kembali berkembang dan bahkan telah menjadi sebuah industry, diawali dengan Digital Equipment Corporation (DEC) penemu sistem yang dinamakan R1 yang digunakan untuk melakukan konfigurasi sistem pada computer baru. Ditahun yang sama, hampir seluruh perusahaan di Amerika Serikat mempunyai divisi sendiri untuk mengembangkan AI. Sehingga pendapatan tahunan sebagian besar perusahaan di Amerika Serikat bertambah hingga \$ 2 Milyar US dollar per tahun.\\
    \item Teknik pengelompokan data\\
    \begin{itemize}
        \item Supervised learning, adalah proses pengelompokan data yang telah memiliki label dan akan dikelompokan berdasarkan labelnya 
        \item unsupervised learning merupakan proses pengelompokan data yang tidak memiliki label.
        \item Regresi merupakan suatu teknik analisis untuk mengidentifikasi relasi atau hubungan diantara dua variable atau lebih
        \item Klasifikasi merupakan teknik untuk mengklasifikasikan atau mengkategorikan beberapa item yang belum berlabel kedalam sebuah set kelas diskrit
    
    \end{itemize}
    \item Data
    \begin{itemize}
        \item Dataset adalah himpunan data yang berasal dari informasi masa-masa lampau dan dikelola menjadi sebuah informasi
        \item Training set adalah bagian dataset yang kita latih untuk membuat prediksi atau menjalankan fungsi dari sebuah algoritma machine learning
        \item Testing set adalah bagian dataset yang kita tes untuk melihat keakuratannya atau dengan kata lain melihat performanya
    \end{itemize}
\end{enumerate}
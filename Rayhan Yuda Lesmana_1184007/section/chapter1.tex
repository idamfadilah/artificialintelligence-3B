\chapter{Mengenal Kecerdasan Buatan dan Scikit-Learn}

\section{Teori}
Praktek teori penunjang yang dikerjakan :
\begin{enumerate}
\item Sejarah Perkembangan dan penjelasan Definisi \textit{Artifical Intelligence}. Artifical Intelligence merupakan suatu ilmu pada bidang komputer yang mana dapat membuat suatu sistem yang cerdas untuk menyelesaikan suatu permasalahan.\\
Pada akhir tahun 1955 istilah dari \textit{Artifical Intelligence} pertama kali diciptakan.Pada awal kemunculnya kecerdasan buatan ini mengalami banyak kesuksesan, yang dimana diawali dengan kesuksesan Newell dan Simon dengan sebuah program yang disebut dengan \textit{The Logic Theorist}, Program ini merepresentasikan  masalah sebagai model pohon, lalu penyelesaiannya dengan  memilih cabang yang akan menghasilkan kesimpulan terbenar. Program ini berdampak besar dan menjadi batu loncatan penting dalam mengembangkan bidang AI. Pada tahun 1956 John McCarthy dari  Massacuhetts Institute of Technology dianggap sebagai bapak AI, menyelenggarakan konferensi untuk menarik para ahli komputer bertemu, dengan  nama kegiatan “The Dartmouth summer research project on artificial intelligence.”. Setelah itu pada tahun 1970, proyek DARPA(\textit{Defence Advanced Research Project Agency})berhasil menyelesaikan studi kasus mengenai pemetaan jalan. Selanjutnya awal abad ke 21 atau lebih tepatnya pada tahun 2003, DARPA juga berhasil menciptakan asisten pribadi cerdas.Sejak itulah kecerdesasan buatan mulai mengalami perkembangan hingga saat ini menjadi program yang sangat kompleks dengan menerapkan algoritma dan \textit{deep learning}. Oleh karena itu kecerdasan buatan mampu memberikan solusi yang kompleks untuk berbagai macam permasalahan dengan inovasi yang bervariatif.

\item Definisi supervised learning, klasifikasi, regresi dan unsupervised learning. Data set, training set dan testing set
\begin{itemize}
    \item Supervised Learning dan Unsupervised Learning
    \par
    \textit{Supervised learning}
    Adalah sebuah pembelajaran yang dapat memprediksi seuatu pola yang mana pola tersebut sudah ada contoh datanya, jadi pola yang sudah ada tadi merupakan hasil dari data yang sudah ada contoh datanya
    \textit{Unsupervised Learning} yaitu tidak menggunakan label dalam memprediksi targetnya melainkan dengan melihat kesamaan dari atribut yang dimiliki. Jika memiliki kesamaan dari atribut atau sifat data yang diextrak maka akan dimasukkan kedalam satu kelompok(\textit{clustering}). Sehingga dapat menimbulkan banyak kelompok.
    \item Klasifikasi dan Regresi
    \par
    Adalah suatu teknik mengklasifikasikan atau mengelompokan beberapa item yang belum ada label kedalam sebuah kelas distrit. Regresi adalah suatu teknik analisis untuk mendifinisikan suatu relasi diantara dua variabel atau lebih, berguna untuk menemukan fungsi model data dengan meminimalkan error atau selisih nilai prediksi dengan nilai sebenarnya.
    
    \item Data set, Training set dan Testing set
    \par
    Adalah suatu objek yang menjelaskan sebuah data dengan ralasinya di memory. Lalu training set adalah bagian dari data set yang berfungsi membuat suatu algoritma sesuai tujuan yang di harapkan, sedangkan testing set yaitu baguan dari data set yang berfungsi untuk melihat ke akuratan dari sebuah data yang di tes.
\end{itemize}
\end{enumerate}

\section{Instalasi}
\begin{enumerate}
\item Melakukan installasi pada anaconda pormt dengan perintah " pip install -U scikit-learn".
    \begin{figure}[!htbp]
    \centering
    \includegraphics[scale=0.4]{figures/install.PNG}
    \end{figure}
    \newpage
    \item Setelah itu masuk ke link website yang telah diberikan yaitu "https://scikit-learn.org/stable/tutorial/basic/tutorial.html".
    \begin{lstlisting}[language=Python]
from sklearn.linear_model import LogisticRegression
from sklearn import set_config


lr = LogisticRegression(penalty='l1')
print('Default representation:')
print(lr)
# LogisticRegression(C=1.0, class_weight=None, dual=False, fit_intercept=True,
#                    intercept_scaling=1, l1_ratio=None, max_iter=100,
#                    multi_class='auto', n_jobs=None, penalty='l1',
#                    random_state=None, solver='warn', tol=0.0001, verbose=0,
#                    warm_start=False)

set_config(print_changed_only=True)
print('\nWith changed_only option:')
print(lr)
# LogisticRegression(penalty='l1')
\end{lstlisting}
\item Mencoba Loading example dataset
 \begin{lstlisting}[language=Python]
from sklearn import datasets 
#mengimport class dataset dari scikit learn library
iris = datasets.load_iris() 
#memuat dan memasukkan dataset iris ke variabel bernama iris
digits = datasets.load_digits() 
#membuat dan memasukkan dataset digits ke variabel digits

print(digits.data) 
#memberikan akses ke fitur yang dapat digunakan untuk mengklasifikasikan  sampel digit dan menampilkan diconsole

digits.target 
#memberikan informasi tentang data yang berhubungan atau juga dapat dijadikan sebagai label

digits.images[0] 
#Data selalu berupa array 2D, shape (n.samples, n.features), meskipun data aslinya mungkin memiliki bentuk yang berbeda.
\end{lstlisting}
\item Mencoba Learning and predicting
\begin{lstlisting}[language=Python]
from sklearn import svm 
#perintah untuk mengimport class svm dari package sklearn

digits = datasets.load_digits()    
#memasukkan dan memuat dataset digits ke variabel digits

clf = svm.SVC(gamma=0.001, C=100.) 
#elf sebagai estimator/parameter, svm.SVC sebagai class, gamma sebagai parameter untuk menerapkan nilai secara manual

clf.fit(digits.data[:-1], digits.target[:-1]) 
#elf sebagai estimator/parameter, fit sebagai metode, digits.data sebagai item,[:-1] sebagai syntax python dan menampilkan outputnya


print(clf.predict(digits.data[-1:])) 
#predict sebagai metode lainnya, digit.data sebagai item menampilkan outputnya
\end{lstlisting}
\item Mencoba Model Persistence
\begin{lstlisting}[language=Python]
from sklearn import svm 
#mengimport class svm dari scikit learn library
from sklearn import datasets 
#mengimport class dataset dari scikit learn library
 
clf = svm.SVC(gamma=0.001, C=100.) 
#memanggil class SVC dan menset argument constructor SVC serta ditampung di variabel clf
X, y= datasets.load_iris(return_X_y=True) 
#meload datasets iris dan ditampung di variabel x untuk data sedangkan y untuk target

clf.fit(X, y) 
#memanggil method fit untuk melakukan training data dengan argumen data dan target dari database iris 

import pickle 
#mengimport pickle (agar dapat terbaca)
s = pickle.dumps(clf) 
#memanggil method dumps dengan argumen clf dan ditampung pada valiabel s
clf2 = pickle.loads(s) 
#memanggil method loads dengan argumen s dan ditampung di variabel clf2
clf2.predict(X[0:1]) 
#menampilkan hasil dari method predict dengan argumen data variabel x 

from joblib import dump, load 
#mengimport dump dan load dari library joblib
dump(clf, '1184007.joblib') 
#memanggil method dumps dengan argumen clf dari nama file joblib
clf3 = load('1184007.joblib') 
#memanggil method load dengan argumen nama file joblibnya
print(clf3.predict(X[0:1])) 
#menampilkan hasil dari method predict dengan argumen data variabel
\end{lstlisting}
\item Mencoba Conventions
\begin{lstlisting}[language=Python]
#Type casting
import numpy as np
from sklearn import random_projection

rng = np.random.RandomState(0)
X = rng.rand(10, 2000)
X = np.array(X, dtype='float32')
print(X.dtype)


transformer = random_projection.GaussianRandomProjection()
X_new = transformer.fit_transform(X)
print(X_new.dtype)

from sklearn import datasets
from sklearn.svm import SVC
iris = datasets.load_iris()
clf = SVC(gamma=0.001, C=100.)
clf.fit(iris.data, iris.target)
print(list(clf.predict(iris.data[:3])))
clf.fit(iris.data, iris.target_names[iris.target])
print(list(clf.predict(iris.data[:3])))

#refitting and updating parameters
import numpy as np
from sklearn.datasets import load_iris
from sklearn.svm import SVC
X, y = load_iris(return_X_y=True)
clf = SVC(gamma=0.001, C=100.)
clf.set_params(kernel='linear').fit(X, y)
clf.set_params(kernel='rbf').fit(X, y)
print(clf.predict(X[:5]))

#multiclass vs multilabel fitting
from sklearn.svm import SVC
from sklearn.multiclass import OneVsRestClassifier
from sklearn.preprocessing import LabelBinarizer

X = [[1, 2], [2, 4], [4, 5], [3, 2], [3, 1]]
y = [0, 0, 1, 1, 2]

classif = OneVsRestClassifier(estimator=SVC(random_state=0, gamma=0.001, C=100.))
print(classif.fit(X, y).predict(X))
y = LabelBinarizer().fit_transform(y)
print(classif.fit(X, y).predict(X))

from sklearn.preprocessing import MultiLabelBinarizer
y = [[0, 1], [0, 2], [1, 3], [0, 2, 3], [2, 4]]
y = MultiLabelBinarizer().fit_transform(y)
print(classif.fit(X, y).predict(X))

\end{lstlisting}
\end{enumerate}


\section{Penanganan Error}
\begin{enumerate}

\item Screenshoot Error
\begin{figure}[!htbp]
    \centering
    \includegraphics[scale=0.5]{figures/identasi.PNG}
    \end{figure}
    \begin{figure}[!htbp]
    \centering
    \includegraphics[scale=0.5]{figures/model.PNG}
    \end{figure}

\newpage    
\item Tuliskan kode error dan jenis error
\par 
IdentationError ( Muncul saat ada indentasi yang salah).\\
ImportError ( Muncul saat modul yang hendak diimpor tidak ditemukan )
\item Solusi dari error tersebut
\par 
Kita harus lebih teliti lagi dan fokus pada saat mencoding agar tidak terjadi identasi dan menuliskan nama file / nama import modul agar tidak terjadi error. 
\end{enumerate}


